\documentclass{scrartcl}
\usepackage{hyperref}
\usepackage{shortvrb}
\usepackage{metalogo}
\usepackage{fontawesome5}
\usepackage[utf8]{inputenc}
\MakeShortVerb{\|}
\begin{document}
\title{The fontawesome5 package\thanks{This document corresponds to fontawesome5 version 5.8.0, dated 2019/03/20}}
\author{Font Awesome\thanks{More information at \url{https://fontawesome.com}} (The font)\and Marcel Krüger\thanks{E-Mail: \href{mailto:tex@2krueger.de}{\nolinkurl{tex@2krueger.de}}} (The \LaTeX{} package)}
\maketitle
This package provides \LaTeX{} support for the Font Awesome 5 icons.

To use Font Awesome 5 icons in your document, include the package with
\begin{verbatim}
  \usepackage{fontawesome5}
\end{verbatim}
Alternatively you can add the |fixed| option to get fixed-width icons:
\begin{verbatim}
  \usepackage[fixed]{fontawesome5}
\end{verbatim}
For every icon a macro is provided: Just use the official icon names\footnote{For a full list see \url{https://fontawesome.com/cheatsheet}.} converted to CamelCase with the prefix |\fa|.
For example to use the |hand-point-up| icon, use |\faHandPointUp|.
For icons ending with |-alt|, append a |*| instead.
An optional argument can be added to select the style (|solid| or |regular|).
The default style is |solid|, it can be changed with |\faStyle{...}|

Every icon can also be accessed with |\faIcon{the-icon-name}| or |\faIcon[style]{...}|.

\subsection*{Example}
\begin{verbatim}
...
\usepackage{fontawesome5}
...
\begin{document}
...
A simple icon: \faHandPointUp\\
Multiple versions of the file icon:
  \faFile~
  \faFile*~
  \faFile[regular]~
  \faFile*[regular].\\
Alternative syntax:
  \faIcon{file}~
  \faIcon*{file}~
  \faIcon[regular]{file}~
  \faIcon*[regular]{file}.
...
\end{document}
\end{verbatim}

A simple icon: \faHandPointUp\\
Multiple versions of the file icon: \faFile~\faFile*~\faFile[regular]~\faFile*[regular].\\
Alternative syntax: \faIcon{file}~\faIcon*{file}~\faIcon[regular]{file}~\faIcon*[regular]{file}.

\subsection*{Font Awesome Pro}
Font Awesome 5 is available in a Free and a  Pro version.
This package uses the free version by default.
If you own a Pro license and have the Font Awesome 5 Pro desktop fonts installed in your system font path, you can use Pro instead.
For this, load the package with the |[pro]| option:
\begin{verbatim}
  \usepackage[pro]{fontawesome5}
\end{verbatim}
Now additional icons and the |light| style can be used.
Pro is only supported with \XeLaTeX{} and \LuaLaTeX.

\subsection*{Updates}
This package corresponds to Font Awesome 5.8.0.
In case there is a newer version available on the Font Awesome homepage, check for updates on \url{https://ctan.org/pkg/fontawesome5}. Should there be no corresponding update on CTAN, you can write a mail to \href{mailto:tex@2krueger.de}{\nolinkurl{tex@2krueger.de}}.
If you use \XeLaTeX{} or \LuaLaTeX{}, you can also directly download the new Desktop Fonts from \url{https://fontawesome.com} into your \TeX{} tree. If you save them with the filenames\\
{\ttfamily\begin{tabular}{l}
  FontAwesome5Brands-Regular-400.otf\\
  FontAwesome5Free-Regular-400.otf\\
  FontAwesome5Free-Solid-900.otf
\end{tabular}}\\
the package will start using the new version right away.

\subsection*{Bugs}
For bug reports and feature requests, write to \href{mailto:tex@2krueger.de}{\nolinkurl{tex@2krueger.de}}.
\end{document}
